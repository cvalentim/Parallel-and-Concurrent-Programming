\documentclass[12pt]{article}

\usepackage[brazilian]{babel}
\usepackage[utf8]{inputenc}

% This first part of the file is called the PREAMBLE. It includes
% customizations and command definitions. The preamble is everything
% between \documentclass and \begin{document}.

\usepackage[margin=1in]{geometry}  % set the margins to 1in on all sides
\usepackage{graphicx}              % to include figures
\usepackage{amsmath}               % great math stuff
\usepackage{amsfonts}              % for blackboard bold, etc
\usepackage{amsthm}                % better theorem environments


% various theorems, numbered by section

\newtheorem{thm}{Theorem}[section]
\newtheorem{lem}[thm]{Lemma}
\newtheorem{prop}[thm]{Proposition}
\newtheorem{cor}[thm]{Corollary}
\newtheorem{conj}[thm]{Conjecture}

\DeclareMathOperator{\id}{id}

\newcommand{\bd}[1]{\mathbf{#1}}  % for bolding symbols
\newcommand{\RR}{\mathbb{R}}      % for Real numbers
\newcommand{\ZZ}{\mathbb{Z}}      % for Integers
\newcommand{\col}[1]{\left[\begin{matrix} #1 \end{matrix} \right]}
\newcommand{\comb}[2]{\binom{#1^2 + #2^2}{#1+#2}}


\begin{document}


\nocite{*}

\title{Árvore Binária de Busca Ótima - Uma Implementação Distribuída}

\author{Felipe Reis e Caio Valentim\\ 
Departamento de Informática \\
PUC-Rio \\ 
}

\maketitle


\section{Introdução}
O problema de encontrar a árvore binária de busca ótima para um conjunto
de frequências é estudado há anos no meio acadêmico. O melhor algoritmo
exato conhecido é uma programação dinâmica de complexidade $O(n^2)$ \cite{knuth71}. 

Contudo, o algoritmo sequencial não é facilmente paralelizável. Existem
algumas propostas, como em \cite{karpinski94} e \cite{mitica}.
Mas, em geral, o apresentado não é uma solução exata ou é complicado demais.

Neste documento apresentamos nosso desenvolvimento e implementação 
de uma versão distribuída simples para o problema.


\section{Definição do Problema}

O problema em questão é o de encontrar uma árvore 
binária de busca ótima para um conjunto de chaves dadas as 
frequências de cada chave. Formalmente,
temos as chaves $A = \{a_1 \le a_2 \le \ldots \le a_n\}$,
as frequências de buscas com sucesso de cada chave $F = \{f_1, f_2, \ldots, f_n\}$ 
e as frequências de buscas sem sucesso $Q = \{q_0, q_1, \ldots, q_n, q_{n+1}\}$.
Onde $q_i$ representa a frequência de buscas por chaves entre $a_i$ e $a_{i+1}$.

Nesse trabalho iremos assumir que todas as buscas são com sucesso. Ou seja,
iremos descartar o conjunto $Q$. Desta forma, queremos criar uma árvore
binária de busca que minimize a seguinte função:

$$ \sum_{i=1}^n f_i \times n(i) $$

onde $n(i)$ é o nível do nó $i$ na árvore.

\section{Artigos Relacionados}

Os dois principais artigos estudados foram \cite{karpinski94} e \cite{mitica}.

No primeiro, M. Karpinski propõe uma solução paralela capaz
de calcular a solução ótima em tempo $O(n^{1-\epsilon})$ para uma
constante arbitrária $0 < \epsilon \le 1/2$. Ele agrupa as diagonais
em conjuntos e calcula cada conjunto de forma eficiente a partir 
do pré-processamento de uma estrutura que ele chama ``sub-árvores especiais''. 
Apesar de interessante, a arbodagem não é fácil de implementar, 
especialmente para quem não possui grande experiência com codificação
em paralelo.

Em outra linha, C. Mitica propõe em \cite{mitica} uma solução paralela
que começa dividindo o conjunto de chaves em intervalos menores,
calculando as árvores ótimas para cada intervalo menor e, a partir
das árvores de cada sub-intervalo, constroi uma solução para todo 
conjunto. Essa solução não foi implementada pois acreditamos que
a abordagem está errada e, de fato, não produz uma árvore ótima.  


\section{Solução Sequencial}

Abaixo dois lemas úteis para construção do algoritmo
sequencial.

\begin{lem}[Critério de Otimalidade]
Seja $OPT(i, j)$ o custo da árvore ótima para as chaves $\{a_i, \ldots, a_{j-1}\}$.
A seguinte recorrência vale:

$$OPT(i, j) = \min_{i \le k < j} \{OPT(i, k) + OPT(k+1, j)\} + \sum_{t=i}^{j-1} f_t$$

\end{lem}

\begin{lem}[Princípio da Monoticidade]
	Seja $r$ a chave que será a raiz da árvore ótima do
	intervalo $[i, j)$ e $[i', j')$ outro intervalo tal que
	$i \le i' \le j \le j'$ com raiz ótima $r'$. Vale
	que $r \le r'$. 
\end{lem}

Com o primeiro resultado é fácil construir uma tabela
com a solução ótima em tempo $O(n^3)$, usando o segundo lema podemos
reduzir a complexidade para $O(n^2)$. Contudo, nossa
versão distribuída se basea na construção $O(n^3)$ implementada
no capítulo 8 do livro \cite{quinn}.

\section{Algoritmo Distribuído}

O algoritmo, derivado do segundo lema,
na prática, consiste em preencher
diagonal por diagonal de uma tabela de dimensões
$n \times n$. Desta forma, como o tempo para calcular cada diagonal
é proporcional a $O(n^2)$, decidimos paralelizar esse cálculo.
Existem $n$ diagonais então, de forma geral, nosso
algoritmos consiste em:

\begin{verbatim}
	for i = 0 to n
		do
			1.    Cada processo calcula um pedaço da i-ésima diagonal
			2.    Cada processo distribui o seu pedaço entre os outros processos
		done
\end{verbatim}

Ou seja, cada processo fica resposável por um pedaço
da diagonal que está sendo calculada no momento. Após o cálculo,
os processos têm que comunicar sua parcela aos outros processos. 
O procedimento se repete até que não existam mais diagonais para calcular.

Com isso, o tempo esperado para o cálculo de cada diagonal é proporcional a
$O(n^2/p + np)$. A primeira parcela da soma se refere ao 
processamento paralelo da diagonal e a segunda ao custo
extra de comunicação.


\section{Implementação}

\subsection{Detalhes do Cluster}
O programa foi rodado no cluster da PUC-Rio que 
conta com 64 nós. Abaixo temos uma descrição técnica 
do cluster retirada do site do suporte do departamento.
A versão do MPI usada foi LAM 7.0.6/MPI 2 C++.

O site da PUC é formado atualmente por três clusters: o primeiro, com 12 estações;
o segundo, com 20 estações; e o terceiro, com 32 estações.
Todos estão situados fisicamente no server farm do DI.
A arquitetura de cada cluster é homogênea: os nós do primeiro cluster têm CPUs Intel Core 2 Duo 2.16 GHz e 1 GB de RAM;
os do segundo têm CPUs Intel Pentium IV 1.70 GHz e 256 Mb de RAM;
e os do terceiro têm CPUs Intel Pentium II 400 MHz (Deschutes) e 280 Mb de RAM.

Os sistemas operacionais instalados são: no primero cluster,
\verb|Fedora Core 8 kernel 2.6.25.4-10|; e, nos demais,
\verb|Red Hat Linux versão 9 kernel 2.4.20-31.9|.
A versão de globus é $2.4$ em todos os clusters.
As versões de MPI disponíveis são \verb|lam-7.1.2|, \verb|lam-7.0.6| e \verb|mpich-1.2.6|. 

O domínio do cluster é par.inf.puc-rio.br.
Os nós estão numerados do n00 até o n63.
O ponto de acesso (via SSH) é a máquina server.par.inf.puc-rio.br.
A partir dela, os nós podem ser acessados via RSH ou SSH. As conexões entre os nós e o switch são de 100Mbs.

Cada usuário possui uma área própria de trabalho em cada nó,
localizada no \verb|/home/local/login_usuário|, 
e acesso remoto a sua área de trabalho na máquina \verb|server.par.inf.puc-rio.br|,
através da pasta \verb|/home/server/login_usuario|.

\subsection{Executando Código MPI}

Uma vez que usamos a implementação LAM do MPI disponível no cluster,
alguns passos foram necessários para conseguir efetivamente executar nosso
código.

Em primeiro lugar, antes de rodar o programa, deve-se criar uma associação
entre um conjunto de máquinas (LAM). Isso é feito com o camando \verb|lamboot -v machines.txt|.
O flag \verb|-v| é de verbose e o arquivo \verb|machines.txt| contém as máquinas que serão
utilizadas. Para o comando funcionar, uma série de requisitos devem ser observados.
Por exemplo, as máquinas devem permitir acesso ssh sem senha. Para
verificar se o ambiente está corretamente configurado, podemos usar o camando \verb| recon -v|
que lista possíveis problemas.

Além do ambiente de execução, precisamos de um código compilado. Para compilar
utilizamos o \verb|mpiCC|, compilador \verb|C++| para código MPI.
Com o código compilado, o próximo passo é distribuir o executável entre as máquinas do cluster.
O suporte do DI disponibiliza o script $mrcp$ para realizar esta tarefa.

Por fim, com todos os passos anteriores feitos, podemos
executar, por exemplo, \verb|mpirun -np 8 a.out|, que roda o programa \verb|a.out|
de forma distribuída entre $8$ processos.


\section{Experimentos}

Os experimentos foram executados com $1, 4$ e $8$ processos rodando em máquinas separadas.
Testamos com entradas de tamanho $10, 50, 100, 500, 1000$ e $5000$.  O tempo 
de execução para cada par entrada x números de processos é a média de $10$
execuções.

Abaixo a tabela dos resultados obtidos:
\begin{center}
\begin{tabular}{c|c|c|c|c|c|c|c}
p & 10 & 50 & 100 & 500 & 1000 & 3000 & 5000 \\
\hline
1 & 0.521 & 0.522 & 0.520 & 0.850 & 3.344 & 107.159 & 503.236 \\
\hline
4 & 0.555 &  0.571 & 0.580 & 0.910 & 2.569 & 63.275 & 295.765 \\
\hline
8 & 0.58 & 0.610 & 0.650 & 1.061 & 2.379 & 38.024 & 172.599 \\
\end{tabular}
\end{center}

E os resultados acima estão dispostos também na figura
~\ref{fig:resultados}.


\clearpage
\begin{figure}
	\begin{center}
		% GNUPLOT: LaTeX picture
\setlength{\unitlength}{0.240900pt}
\ifx\plotpoint\undefined\newsavebox{\plotpoint}\fi
\begin{picture}(1500,900)(0,0)
\sbox{\plotpoint}{\rule[-0.200pt]{0.400pt}{0.400pt}}%
\put(130.0,131.0){\rule[-0.200pt]{4.818pt}{0.400pt}}
\put(110,131){\makebox(0,0)[r]{ 0}}
\put(1419.0,131.0){\rule[-0.200pt]{4.818pt}{0.400pt}}
\put(130.0,239.0){\rule[-0.200pt]{4.818pt}{0.400pt}}
\put(110,239){\makebox(0,0)[r]{ 100}}
\put(1419.0,239.0){\rule[-0.200pt]{4.818pt}{0.400pt}}
\put(130.0,346.0){\rule[-0.200pt]{4.818pt}{0.400pt}}
\put(110,346){\makebox(0,0)[r]{ 200}}
\put(1419.0,346.0){\rule[-0.200pt]{4.818pt}{0.400pt}}
\put(130.0,454.0){\rule[-0.200pt]{4.818pt}{0.400pt}}
\put(110,454){\makebox(0,0)[r]{ 300}}
\put(1419.0,454.0){\rule[-0.200pt]{4.818pt}{0.400pt}}
\put(130.0,561.0){\rule[-0.200pt]{4.818pt}{0.400pt}}
\put(110,561){\makebox(0,0)[r]{ 400}}
\put(1419.0,561.0){\rule[-0.200pt]{4.818pt}{0.400pt}}
\put(130.0,669.0){\rule[-0.200pt]{4.818pt}{0.400pt}}
\put(110,669){\makebox(0,0)[r]{ 500}}
\put(1419.0,669.0){\rule[-0.200pt]{4.818pt}{0.400pt}}
\put(130.0,776.0){\rule[-0.200pt]{4.818pt}{0.400pt}}
\put(110,776){\makebox(0,0)[r]{ 600}}
\put(1419.0,776.0){\rule[-0.200pt]{4.818pt}{0.400pt}}
\put(130.0,131.0){\rule[-0.200pt]{0.400pt}{4.818pt}}
\put(130,90){\makebox(0,0){ 0}}
\put(130.0,756.0){\rule[-0.200pt]{0.400pt}{4.818pt}}
\put(261.0,131.0){\rule[-0.200pt]{0.400pt}{4.818pt}}
\put(261,90){\makebox(0,0){ 500}}
\put(261.0,756.0){\rule[-0.200pt]{0.400pt}{4.818pt}}
\put(392.0,131.0){\rule[-0.200pt]{0.400pt}{4.818pt}}
\put(392,90){\makebox(0,0){ 1000}}
\put(392.0,756.0){\rule[-0.200pt]{0.400pt}{4.818pt}}
\put(523.0,131.0){\rule[-0.200pt]{0.400pt}{4.818pt}}
\put(523,90){\makebox(0,0){ 1500}}
\put(523.0,756.0){\rule[-0.200pt]{0.400pt}{4.818pt}}
\put(654.0,131.0){\rule[-0.200pt]{0.400pt}{4.818pt}}
\put(654,90){\makebox(0,0){ 2000}}
\put(654.0,756.0){\rule[-0.200pt]{0.400pt}{4.818pt}}
\put(784.0,131.0){\rule[-0.200pt]{0.400pt}{4.818pt}}
\put(784,90){\makebox(0,0){ 2500}}
\put(784.0,756.0){\rule[-0.200pt]{0.400pt}{4.818pt}}
\put(915.0,131.0){\rule[-0.200pt]{0.400pt}{4.818pt}}
\put(915,90){\makebox(0,0){ 3000}}
\put(915.0,756.0){\rule[-0.200pt]{0.400pt}{4.818pt}}
\put(1046.0,131.0){\rule[-0.200pt]{0.400pt}{4.818pt}}
\put(1046,90){\makebox(0,0){ 3500}}
\put(1046.0,756.0){\rule[-0.200pt]{0.400pt}{4.818pt}}
\put(1177.0,131.0){\rule[-0.200pt]{0.400pt}{4.818pt}}
\put(1177,90){\makebox(0,0){ 4000}}
\put(1177.0,756.0){\rule[-0.200pt]{0.400pt}{4.818pt}}
\put(1308.0,131.0){\rule[-0.200pt]{0.400pt}{4.818pt}}
\put(1308,90){\makebox(0,0){ 4500}}
\put(1308.0,756.0){\rule[-0.200pt]{0.400pt}{4.818pt}}
\put(1439.0,131.0){\rule[-0.200pt]{0.400pt}{4.818pt}}
\put(1439,90){\makebox(0,0){ 5000}}
\put(1439.0,756.0){\rule[-0.200pt]{0.400pt}{4.818pt}}
\put(130.0,131.0){\rule[-0.200pt]{0.400pt}{155.380pt}}
\put(130.0,131.0){\rule[-0.200pt]{315.338pt}{0.400pt}}
\put(1439.0,131.0){\rule[-0.200pt]{0.400pt}{155.380pt}}
\put(130.0,776.0){\rule[-0.200pt]{315.338pt}{0.400pt}}
\put(784,29){\makebox(0,0){Numero de chaves}}
\put(784,838){\makebox(0,0){Tempo x Tamanho entrada}}
\put(1279,736){\makebox(0,0)[r]{1 processo}}
\put(1299.0,736.0){\rule[-0.200pt]{24.090pt}{0.400pt}}
\put(133,132){\usebox{\plotpoint}}
\multiput(261.00,132.61)(29.040,0.447){3}{\rule{17.567pt}{0.108pt}}
\multiput(261.00,131.17)(94.540,3.000){2}{\rule{8.783pt}{0.400pt}}
\multiput(392.00,135.58)(2.361,0.499){219}{\rule{1.985pt}{0.120pt}}
\multiput(392.00,134.17)(518.881,111.000){2}{\rule{0.992pt}{0.400pt}}
\multiput(915.00,246.58)(0.615,0.500){849}{\rule{0.592pt}{0.120pt}}
\multiput(915.00,245.17)(522.771,426.000){2}{\rule{0.296pt}{0.400pt}}
\put(133,132){\makebox(0,0){$+$}}
\put(143,132){\makebox(0,0){$+$}}
\put(156,132){\makebox(0,0){$+$}}
\put(261,132){\makebox(0,0){$+$}}
\put(392,135){\makebox(0,0){$+$}}
\put(915,246){\makebox(0,0){$+$}}
\put(1439,672){\makebox(0,0){$+$}}
\put(1349,736){\makebox(0,0){$+$}}
\put(133.0,132.0){\rule[-0.200pt]{30.835pt}{0.400pt}}
\put(1279,695){\makebox(0,0)[r]{4 processos}}
\multiput(1299,695)(20.756,0.000){5}{\usebox{\plotpoint}}
\put(1399,695){\usebox{\plotpoint}}
\put(133,132){\usebox{\plotpoint}}
\put(133.00,132.00){\usebox{\plotpoint}}
\put(153.76,132.00){\usebox{\plotpoint}}
\multiput(156,132)(20.756,0.000){5}{\usebox{\plotpoint}}
\multiput(261,132)(20.753,0.317){6}{\usebox{\plotpoint}}
\multiput(392,134)(20.597,2.560){25}{\usebox{\plotpoint}}
\multiput(915,199)(18.733,8.937){28}{\usebox{\plotpoint}}
\put(1439,449){\usebox{\plotpoint}}
\put(133,132){\makebox(0,0){$\times$}}
\put(143,132){\makebox(0,0){$\times$}}
\put(156,132){\makebox(0,0){$\times$}}
\put(261,132){\makebox(0,0){$\times$}}
\put(392,134){\makebox(0,0){$\times$}}
\put(915,199){\makebox(0,0){$\times$}}
\put(1439,449){\makebox(0,0){$\times$}}
\put(1349,695){\makebox(0,0){$\times$}}
\sbox{\plotpoint}{\rule[-0.400pt]{0.800pt}{0.800pt}}%
\sbox{\plotpoint}{\rule[-0.200pt]{0.400pt}{0.400pt}}%
\put(1279,654){\makebox(0,0)[r]{8 processos}}
\sbox{\plotpoint}{\rule[-0.400pt]{0.800pt}{0.800pt}}%
\put(1299.0,654.0){\rule[-0.400pt]{24.090pt}{0.800pt}}
\put(133,132){\usebox{\plotpoint}}
\put(261,131.34){\rule{31.558pt}{0.800pt}}
\multiput(261.00,130.34)(65.500,2.000){2}{\rule{15.779pt}{0.800pt}}
\multiput(392.00,135.41)(7.007,0.503){69}{\rule{11.211pt}{0.121pt}}
\multiput(392.00,132.34)(499.732,38.000){2}{\rule{5.605pt}{0.800pt}}
\multiput(915.00,173.41)(1.813,0.501){283}{\rule{3.091pt}{0.121pt}}
\multiput(915.00,170.34)(517.584,145.000){2}{\rule{1.546pt}{0.800pt}}
\put(133,132){\makebox(0,0){$\ast$}}
\put(143,132){\makebox(0,0){$\ast$}}
\put(156,132){\makebox(0,0){$\ast$}}
\put(261,132){\makebox(0,0){$\ast$}}
\put(392,134){\makebox(0,0){$\ast$}}
\put(915,172){\makebox(0,0){$\ast$}}
\put(1439,317){\makebox(0,0){$\ast$}}
\put(1349,654){\makebox(0,0){$\ast$}}
\put(133.0,132.0){\rule[-0.400pt]{30.835pt}{0.800pt}}
\sbox{\plotpoint}{\rule[-0.200pt]{0.400pt}{0.400pt}}%
\put(130.0,131.0){\rule[-0.200pt]{0.400pt}{155.380pt}}
\put(130.0,131.0){\rule[-0.200pt]{315.338pt}{0.400pt}}
\put(1439.0,131.0){\rule[-0.200pt]{0.400pt}{155.380pt}}
\put(130.0,776.0){\rule[-0.200pt]{315.338pt}{0.400pt}}
\end{picture}

	\end{center}
\caption{Tempos de Execução}	
\label{fig:resultados}
\end{figure}


\section{Conclusão}

De acordo com os experimentos realizados, é possível notar uma melhora significativa 
no tempo de execução total para entradas superiores a 1000 chaves, como esperávamos 
alcançar utilizando computação distribuída. 

Analisando o teste de 5000 chaves, por exemplo, temos um ganho de tempo de $291,5\%$, 
no comparativo entre a execução sequencial e a execução distribuída em $8$ processos.

Outro resultado que também é possível identificar através dos experimentos realizados, 
é o ônus introduzido pela comunicação entre processo, necessária para mantê-los 
sincronizados, na versão distribuída do algoritmo. Se observarmos a execução do algoritmo 
para um conjunto de $500$ chaves, por exemplo, percebemos que a execução entre $8$ processos 
é cerca de $11\%$ mais custosa do que a execução para $4$ processos, e $20\%$ mais ineficiente do 
que a versão paralela.

Em suma, o algoritmo distribuído, para o problema de encontrar a árvore de binária de busca
ótima, apresentou-se bem mais eficiente, se comparado a sua versão sequencial para um conjunto
grande de chaves. No entanto, executando-o com poucas chaves, sua utilização não compensa o tempo 
gasto com comunicação e sincronização entre processos.

\bibliographystyle{plain}

\bibliography{paper}

\end{document}
